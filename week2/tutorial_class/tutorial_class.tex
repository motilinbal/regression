\documentclass[11pt, a4paper]{article}

% --- Packages ---
\usepackage[margin=1in]{geometry} % Sensible margins
\usepackage{amsmath}             % Essential math tools
\usepackage{amssymb}             % More math symbols (\mathbb{R}, etc.)
\usepackage{amsthm}              % Theorem environments
\usepackage{hyperref}            % Clickable links (optional)
\usepackage{lmodern}             % Use a modern font that looks good on screen
\usepackage[T1]{fontenc}         % Support for characters
\usepackage{mathtools}           % Enhancements for amsmath

% --- Hyperref Setup (Optional) ---
\hypersetup{
    colorlinks=true,
    linkcolor=blue,
    filecolor=magenta,      
    urlcolor=cyan,
    pdftitle={Linear Algebra Lecture Notes},
    pdfpagemode=FullScreen,
}

% --- Theorem Styles ---
\newtheoremstyle{mytheoremstyle}
  {\topsep} % Space above
  {\topsep} % Space below
  {\itshape} % Body font
  {} % Indent amount
  {\bfseries} % Theorem head font
  {.} % Punctuation after theorem head
  {.5em} % Space after theorem head
  {} % Theorem head spec (can be left empty, meaning `normal')

\newtheoremstyle{mydefstyle}
  {\topsep} % Space above
  {\topsep} % Space below
  {\normalfont} % Body font
  {} % Indent amount
  {\bfseries} % Theorem head font
  {.} % Punctuation after theorem head
  {.5em} % Space after theorem head
  {} % Theorem head spec

\theoremstyle{mytheoremstyle}
\newtheorem{theorem}{Theorem}[section]
\newtheorem{lemma}[theorem]{Lemma}
\newtheorem{proposition}[theorem]{Proposition}
\newtheorem{corollary}[theorem]{Corollary}
\newtheorem{claim}{Claim}

\theoremstyle{mydefstyle}
\newtheorem{definition}[theorem]{Definition}
\newtheorem{example}[theorem]{Example}
\newtheorem{remark}[theorem]{Remark}

% --- Math Macros ---
\newcommand{\R}{\mathbb{R}}      % Real numbers
\newcommand{\N}{\mathbb{N}}      % Natural numbers
% Using V and W directly in text now for vector spaces
\newcommand{\B}{\mathcal{B}}    % Basis B
\newcommand{\C}{\mathcal{C}}    % Basis C
\newcommand{\E}{\mathcal{E}}    % Standard Basis E
\newcommand{\vb}{\mathbf{v}}     % Vector v
\newcommand{\wb}{\mathbf{w}}     % Vector w
\newcommand{\ub}{\mathbf{u}}     % Vector u
\newcommand{\xb}{\mathbf{x}}     % Vector x
\newcommand{\yb}{\mathbf{y}}     % Vector y
\newcommand{\eb}{\mathbf{e}}     % Standard basis vector e
\newcommand{\bb}{\mathbf{b}}     % Basis vector b
\newcommand{\cb}{\mathbf{c}}     % Basis vector c
\newcommand{\kb}{\mathbf{k}}     % Coordinate vector k
\newcommand{\zerob}{\mathbf{0}}  % Zero vector
\DeclareMathOperator{\Ker}{Ker}  % Kernel
\DeclareMathOperator{\Img}{Im}   % Image (Range)
\DeclareMathOperator{\Span}{Span}% Span
\DeclareMathOperator{\rank}{rank}% Rank
\DeclareMathOperator{\nullity}{nullity} % Nullity
\DeclareMathOperator{\tr}{tr}    % Trace <--- THIS LINE IS REINSTATED
% \det is usually predefined by amsmath
\newcommand{\coord}[2]{[\,#1\,]_{\mathcal{#2}}} % Coordinate vector [v]_B
\newcommand{\matrixrep}[3]{[\,#1\,]_{\mathcal{#2} \leftarrow \mathcal{#3}}} % Matrix rep [T]_C<-B
\newcommand{\changebasis}[2]{[I]_{\mathcal{#1} \leftarrow \mathcal{#2}}} % Change of Basis [I]_C<-B

% --- Document Start ---
\begin{document}

\title{Linear Algebra Notes: \\ Transformations, Representation, and Diagonalization}
\author{Insights from our Lecture}
\date{\today} % Or specify a date
\maketitle

\begin{abstract}
These notes revisit key concepts from our recent linear algebra lecture. We explore the fundamental nature of linear transformations, the utility of representing them as matrices, the mechanics and meaning of changing bases, and the powerful technique of diagonalization, culminating in its geometric interpretation. Our aim is to build intuition and see the interconnectedness of these ideas.
\end{abstract}

\section{Introduction}

We began our session acknowledging that grappling with new concepts, like those in the recent exercises, is part of the learning process. Today, we aim to solidify our understanding of several core topics:
\begin{itemize}
    \item Linear Transformations: The building blocks of linear structure preservation.
    \item Matrix Representation: How to capture the action of a transformation in a convenient numerical form.
    \item Change of Basis: The art of choosing the right perspective (coordinate system) for a problem.
    \item Diagonalization: A technique to simplify transformations by finding a special basis (eigenvectors) where the transformation acts simply as scaling.
    \item Geometric Interpretation: Visualizing the action of matrices, especially symmetric ones, as combinations of rotations, reflections, and scaling.
\end{itemize}
We also hope to touch upon Projection Matrices, though we might need to dedicate more time to them during the upcoming Passover break. Let's dive in!

\section{Linear Transformations: The Essence}

At the heart of linear algebra are functions that respect the underlying vector space structure (addition and scalar multiplication).

\begin{definition}[Linear Transformation]
Let $V$ and $W$ be vector spaces. A function $T: V \to W$ is called a \textbf{linear transformation} if for all vectors $\ub, \vb \in V$ and all scalars $\alpha \in \R$:
\begin{enumerate}
    \item $T(\ub + \vb) = T(\ub) + T(\vb)$ (Additivity)
    \item $T(\alpha \vb) = \alpha T(\vb)$ (Homogeneity)
\end{enumerate}
Equivalently, a single condition encapsulates both: for all $\vb_1, \dots, \vb_n \in V$ and scalars $\alpha_1, \dots, \alpha_n \in \R$,
\[ T(\alpha_1 \vb_1 + \dots + \alpha_n \vb_n) = \alpha_1 T(\vb_1) + \dots + \alpha_n T(\vb_n) \quad \text{or} \quad T\left(\sum_{i=1}^n \alpha_i \vb_i\right) = \sum_{i=1}^n \alpha_i T(\vb_i). \]
\end{definition}

Two crucial subspaces associated with any linear transformation $T: V \to W$ tell us about its behavior:

\begin{definition}[Kernel and Image]
\leavevmode
\begin{itemize}
    \item The \textbf{Kernel} (or Null Space) of $T$, denoted $\Ker(T)$, is the set of vectors in the domain $V$ that are mapped to the zero vector $\zerob_W$ in the codomain $W$:
    \[ \Ker(T) = \{ \vb \in V \mid T(\vb) = \zerob_W \}. \]
    (\textit{Think: What gets "squashed" to zero?})
    \item The \textbf{Image} (or Range) of $T$, denoted $\Img(T)$, is the set of all possible output vectors in the codomain $W$:
    \[ \Img(T) = \{ \wb \in W \mid \exists \vb \in V \text{ such that } T(\vb) = \wb \}. \]
    (\textit{Think: What vectors can we actually "reach" in W?})
\end{itemize}
\end{definition}

\begin{remark}
Both $\Ker(T)$ and $\Img(T)$ are subspaces --- the kernel is a subspace of the domain $V$, and the image is a subspace of the codomain $W$. This is a direct consequence of the linearity of $T$.
\end{remark}

\section{Matrices and Transformations: A Concrete Link}

Matrices provide a powerful way to compute and analyze linear transformations, especially between finite-dimensional spaces like $\R^n$.

\subsection{Matrix-Vector Multiplication as Linear Combination}

A fundamental insight connects matrix multiplication to the structure of the matrix itself.

\begin{proposition}
Let $A$ be an $n \times m$ matrix with columns $\mathbf{a}_1, \mathbf{a}_2, \dots, \mathbf{a}_m \in \R^n$. Let $\ub = [u_1, u_2, \dots, u_m]^T \in \R^m$. The matrix-vector product $\vb = A\ub$ is the linear combination of the columns of $A$ with coefficients given by the entries of $\ub$:
\[ \vb = A\ub = u_1 \mathbf{a}_1 + u_2 \mathbf{a}_2 + \dots + u_m \mathbf{a}_m = \sum_{j=1}^m u_j \mathbf{a}_j. \]
\end{proposition}

\begin{corollary}
The result $\vb = A\ub$ is always in the Column Space of $A$, i.e., $\vb \in \operatorname{Col}(A) = \Span\{\mathbf{a}_1, \dots, \mathbf{a}_m\}$.
\end{corollary}

\begin{remark}
This viewpoint is incredibly useful. It tells us that the transformation $T(\ub) = A\ub$ maps vectors from $\R^m$ into the column space of $A$ (which is a subspace of $\R^n$). The Image of this transformation *is* the column space: $\Img(T) = \operatorname{Col}(A)$. Similarly, the Kernel of $T$ is the Null Space of $A$: $\Ker(T) = \operatorname{Nul}(A) = \{\ub \in \R^m \mid A\ub = \zerob\}$.
\end{remark}

\begin{theorem}[Rank-Nullity Theorem]
Let $T: V \to W$ be a linear transformation where $V$ is a finite-dimensional vector space. Then the dimensions of the kernel and image are related by:
\[ \dim(\Ker(T)) + \dim(\Img(T)) = \dim(V). \]
When considering the matrix transformation $T(\xb) = A\xb$ where $A$ is $m \times n$:
\[ \nullity(A) + \rank(A) = n, \]
where $\nullity(A) = \dim(\operatorname{Nul}(A))$ and $\rank(A) = \dim(\operatorname{Col}(A))$.
\end{theorem}

\begin{remark}
This fundamental theorem connects the dimension of the space that gets "lost" (Kernel) and the dimension of the space that is "reached" (Image) to the dimension of the original input space.
\end{remark}


\subsection{Coordinate Vectors: Describing Vectors Relative to a Basis}

While vectors exist independently, we often describe them using coordinates relative to a chosen basis. Think of a basis as defining a "coordinate system" or a "language" for the vector space.

\begin{definition}[Coordinate Vector]
Let $\B = \{\bb_1, \bb_2, \dots, \bb_n\}$ be an \textbf{ordered basis} for a vector space $V$. Any vector $\vb \in V$ can be written \emph{uniquely} as a linear combination of the basis vectors:
\[ \vb = k_1 \bb_1 + k_2 \bb_2 + \dots + k_n \bb_n = \sum_{i=1}^n k_i \bb_i. \]
The vector $\kb = [k_1, k_2, \dots, k_n]^T \in \R^n$ is called the \textbf{coordinate vector of $\vb$ relative to the basis $\B$}, denoted $\coord{\vb}{B}$.
\end{definition}

\begin{example}[Standard Basis]
Let $\E = \{\eb_1, \dots, \eb_n\}$ be the standard basis for $\R^n$. For any vector $\vb = [v_1, \dots, v_n]^T \in \R^n$, we have $\vb = v_1 \eb_1 + \dots + v_n \eb_n$. Thus, the coordinate vector relative to the standard basis is the vector itself:
\[ \coord{\vb}{E} = \vb. \]
The standard basis provides the "natural" coordinate system we usually work with.
\end{example}

\begin{example}[A Non-Standard Basis in $\R^2$]
Let $\B = \{\bb_1, \bb_2\} = \{[1, 1]^T, [1, 0]^T\}$. This is a basis for $\R^2$. Consider the vector $\vb = [2, 5]^T$. What is $\coord{\vb}{B}$? We seek scalars $k_1, k_2$ such that
\[ k_1 \begin{pmatrix} 1 \\ 1 \end{pmatrix} + k_2 \begin{pmatrix} 1 \\ 0 \end{pmatrix} = \begin{pmatrix} 2 \\ 5 \end{pmatrix}. \]
This yields the system:
\begin{align*} k_1 + k_2 &= 2 \\ k_1 \qquad &= 5 \end{align*}
Solving gives $k_1 = 5$ and $k_2 = 2 - 5 = -3$. Therefore,
\[ \coord{\vb}{B} = \begin{pmatrix} 5 \\ -3 \end{pmatrix}. \]
This means $\vb = 5\bb_1 - 3\bb_2$. Geometrically, the vector $\vb$ is the same point, but we describe how to reach it using steps along the $\B$-basis vectors instead of the standard axes.
\end{example}

\begin{remark}
If we form a matrix $B = [\bb_1 \dots \bb_n]$ whose columns are the basis vectors, the equation $\vb = \sum k_i \bb_i$ becomes the matrix equation $\vb = B \coord{\vb}{B}$. Since the columns of $B$ form a basis for $\R^n$, $B$ is invertible, and we can find the coordinates via:
\[ \coord{\vb}{B} = B^{-1} \vb. \]
\end{remark}

\section{Changing Perspective: Change of Basis}

Often, a problem becomes much simpler if viewed in a different coordinate system (basis). How do we translate between these different perspectives?

\begin{definition}[Change-of-Coordinates Matrix]
Let $\B = \{\bb_1, \dots, \bb_n\}$ and $\C = \{\cb_1, \dots, \cb_n\}$ be two bases for a vector space $V$. The \textbf{change-of-coordinates matrix from $\B$ to $\C$}, denoted $\changebasis{C}{B}$, is the unique $n \times n$ matrix such that for any vector $\vb \in V$:
\[ \coord{\vb}{C} = \changebasis{C}{B} \coord{\vb}{B}. \]
This matrix converts $\B$-coordinates into $\C$-coordinates.
\end{definition}

\begin{proposition}[Constructing the Change-of-Basis Matrix]
The columns of $\changebasis{C}{B}$ are the $\C$-coordinate vectors of the $\B$-basis vectors:
\[ \changebasis{C}{B} = \begin{bmatrix} \coord{\bb_1}{C} & \coord{\bb_2}{C} & \dots & \coord{\bb_n}{C} \end{bmatrix}. \]
If working in $\R^n$, let $B = [\bb_1 \dots \bb_n]$ and $C = [\cb_1 \dots \cb_n]$ be the matrices whose columns are the basis vectors. Then
\[ \changebasis{C}{B} = C^{-1} B. \]
\end{proposition}

\begin{proof}[Sketch]
We know $\vb = B \coord{\vb}{B}$ and $\vb = C \coord{\vb}{C}$.
So, $C \coord{\vb}{C} = B \coord{\vb}{B}$. Multiplying by $C^{-1}$ gives $\coord{\vb}{C} = (C^{-1} B) \coord{\vb}{B}$. By uniqueness, $\changebasis{C}{B} = C^{-1} B$.
\end{proof}

\begin{remark}[Inverse Property]
Changing from $\C$ to $\B$ is the inverse operation:
\[ \changebasis{B}{C} = (\changebasis{C}{B})^{-1}. \]
\end{remark}

\begin{remark}[Standard Basis Connection]
Let $\E$ be the standard basis in $\R^n$.
\begin{itemize}
    \item $\changebasis{E}{B} = B$ (Matrix $B$ converts $\B$-coordinates to standard coordinates).
    \item $\changebasis{B}{E} = B^{-1}$ (Matrix $B^{-1}$ converts standard coordinates to $\B$-coordinates).
\end{itemize}
\end{remark}

\section{Matrix Representation of Linear Transformations}

Just as we represent vectors using coordinates, we can represent linear transformations using matrices, provided we fix bases for the domain and codomain.

\begin{definition}[Matrix for a Linear Transformation]
Let $T: V \to W$ be a linear transformation. Let $\B = \{\bb_1, \dots, \bb_n\}$ be a basis for $V$ and $\C = \{\cb_1, \dots, \cb_m\}$ be a basis for $W$. The \textbf{matrix for $T$ relative to bases $\B$ and $\C$} is the unique $m \times n$ matrix $\matrixrep{T}{C}{B}$ such that for every $\vb \in V$:
\[ \coord{T(\vb)}{C} = \matrixrep{T}{C}{B} \coord{\vb}{B}. \]
This matrix transforms the $\B$-coordinates of the input vector into the $\C$-coordinates of the output vector.
\end{definition}

\begin{remark}[Construction]
The $j$-th column of $\matrixrep{T}{C}{B}$ is the $\C$-coordinate vector of the image of the $j$-th basis vector of $\B$:
\[ \text{Column } j \text{ of } \matrixrep{T}{C}{B} = \coord{T(\bb_j)}{C}. \]
\end{remark}

\begin{remark}[Identity Transformation]
If $T: V \to V$ is the identity transformation ($T(\vb) = \vb$), then its matrix representation relative to bases $\B$ and $\C$ is simply the change-of-basis matrix:
\[ \matrixrep{Id}{C}{B} = \changebasis{C}{B}. \]
\end{remark}

\begin{remark}[Standard Representation]
For a transformation $T(\xb) = A\xb$ from $\R^n$ to $\R^m$, the matrix $A$ *is* the matrix representation relative to the standard bases $\E_n$ and $\E_m$:
\[ \matrixrep{T}{E_m}{E_n} = A. \]
This is why we often seamlessly switch between thinking about a matrix $A$ and the transformation $T(\xb)=A\xb$.
\end{remark}

\subsection{How Representation Changes with Basis}

A key question arises: If we know the matrix for $T$ in one basis, how do we find it in another? This links matrix representation with change of basis.

\begin{theorem}[Change of Basis for Transformation Matrices]
Let $T: V \to V$ be a linear transformation, and let $\B$ and $\C$ be two bases for $V$. Then the matrix representations of $T$ relative to these bases (i.e., $\matrixrep{T}{B}{B}$ and $\matrixrep{T}{C}{C}$) are related by:
\[ \matrixrep{T}{C}{C} = \changebasis{C}{B} \matrixrep{T}{B}{B} \changebasis{B}{C}. \]
(Recall that $\changebasis{B}{C} = (\changebasis{C}{B})^{-1}$.)
\end{theorem}

\begin{proof}[Idea]
We want to compute $\coord{T(\vb)}{C}$.
\begin{enumerate}
    \item Start with $\coord{\vb}{C}$.
    \item Convert to $\B$-coordinates: $\coord{\vb}{B} = \changebasis{B}{C} \coord{\vb}{C}$.
    \item Apply $T$ using the $\B$-basis representation: $\coord{T(\vb)}{B} = \matrixrep{T}{B}{B} \coord{\vb}{B} = \matrixrep{T}{B}{B} \changebasis{B}{C} \coord{\vb}{C}$.
    \item Convert the result back to $\C$-coordinates: $\coord{T(\vb)}{C} = \changebasis{C}{B} \coord{T(\vb)}{B} = \changebasis{C}{B} \matrixrep{T}{B}{B} \changebasis{B}{C} \coord{\vb}{C}$.
\end{enumerate}
Since this holds for all $\coord{\vb}{C}$, the matrix must be $\matrixrep{T}{C}{C} = \changebasis{C}{B} \matrixrep{T}{B}{B} \changebasis{B}{C}$.
\end{proof}

\begin{remark}[Similarity]
Matrices $M$ and $N$ are called \textbf{similar} if $M = P N P^{-1}$ for some invertible matrix $P$. The theorem shows that matrices representing the *same* linear transformation $T$ but with respect to *different* bases are similar. $P$ here is the change-of-basis matrix $\changebasis{C}{B}$. This suggests we might want to find a basis $\B$ where the matrix $\matrixrep{T}{B}{B}$ is particularly simple (e.g., diagonal)!
\end{remark}

\section{Diagonalization: Finding the Simplest Perspective}

Diagonalization is the process of finding a basis in which the matrix representation of a transformation becomes diagonal. This simplifies the transformation's action to simple scaling along the basis directions.

\begin{definition}[Diagonalizable Matrix]
An $n \times n$ matrix $A$ is \textbf{diagonalizable} if it is similar to a diagonal matrix $D$. That is, if there exists an invertible matrix $P$ and a diagonal matrix $D$ such that
\[ A = P D P^{-1}. \]
\end{definition}

\begin{theorem}[Diagonalization Theorem]
An $n \times n$ matrix $A$ is diagonalizable if and only if $A$ has $n$ linearly independent eigenvectors.
If $A = P D P^{-1}$, then:
\begin{itemize}
    \item The columns of $P$ are $n$ linearly independent eigenvectors of $A$.
    \item The diagonal entries of $D$ are the eigenvalues of $A$, corresponding to the eigenvectors in the columns of $P$ (in the same order).
\end{itemize}
\end{theorem}

\begin{remark}
Recall that an eigenvector $\vb$ of $A$ with eigenvalue $\lambda$ satisfies $A\vb = \lambda\vb$. Eigenvectors point in directions that are only scaled (not changed in direction) by the transformation $A$.
\end{remark}

\begin{theorem}[Symmetric Matrices]
If an $n \times n$ matrix $A$ is symmetric ($A = A^T$), then it is always \textbf{orthogonally diagonalizable}. This means there exists an orthogonal matrix $U$ ($U^{-1} = U^T$) and a diagonal matrix $\Lambda$ such that
\[ A = U \Lambda U^T. \]
The columns of $U$ form an orthonormal basis for $\R^n$ consisting of eigenvectors of $A$, and the diagonal entries of $\Lambda$ are the corresponding real eigenvalues.
\end{theorem}

\subsection{Diagonalization as Change of Basis}

Let's connect diagonalization back to matrix representation. Consider the transformation $T(\xb) = A\xb$. If $A$ is diagonalizable, let $\B$ be the basis formed by the linearly independent eigenvectors (columns of $P$, or $U$ in the symmetric case).

\begin{proposition}
If $A = P D P^{-1}$, let $\B$ be the basis of eigenvectors forming the columns of $P$. Then the matrix representation of the transformation $T(\xb)=A\xb$ with respect to the basis $\B$ is the diagonal matrix $D$:
\[ \matrixrep{T}{B}{B} = D. \]
Similarly, if $A = U \Lambda U^T$ (symmetric case), let $\B$ be the orthonormal basis of eigenvectors forming the columns of $U$. Then:
\[ \matrixrep{T}{B}{B} = \Lambda. \]
\end{proposition}

\begin{proof}[Symmetric Case]
We want $\matrixrep{T}{B}{B}$. Using the change of basis formula from standard basis $\E$ to $\B$:
\[ \matrixrep{T}{B}{B} = \changebasis{B}{E} \matrixrep{T}{E}{E} \changebasis{E}{B}. \]
Here:
\begin{itemize}
    \item $\matrixrep{T}{E}{E} = A$.
    \item $\changebasis{E}{B} = U$ (columns are the new basis vectors).
    \item $\changebasis{B}{E} = U^{-1} = U^T$ (since $U$ is orthogonal).
\end{itemize}
Substituting:
\[ \matrixrep{T}{B}{B} = U^T A U = U^T (U \Lambda U^T) U = (U^T U) \Lambda (U^T U) = I \Lambda I = \Lambda. \]
\end{proof}

\textbf{The Punchline:} Diagonalizing $A$ is equivalent to finding a basis (the eigenvector basis) where the transformation $T(\xb)=A\xb$ acts in the simplest possible way: just scaling along the basis axes by the eigenvalues. The equation $A = P D P^{-1}$ (or $A = U \Lambda U^T$) precisely describes the process:
\begin{enumerate}
    \item $P^{-1}$ (or $U^T$): Change from standard coordinates to the eigenvector basis coordinates.
    \item $D$ (or $\Lambda$): Apply the simple scaling transformation in this eigenvector basis.
    \item $P$ (or $U$): Change back from the eigenvector basis coordinates to standard coordinates.
\end{enumerate}

\section{Geometric Interpretation of Symmetric Transformations}

For a symmetric matrix $A = U \Lambda U^T$, what does the transformation $T(\xb) = A\xb$ look like geometrically?

The action $\yb = U \Lambda U^T \xb$ can be seen as a sequence of three steps:

\begin{enumerate}
    \item \textbf{Rotation/Reflection ($\xb' = U^T \xb$):} Since $U^T$ is an orthogonal matrix, it preserves lengths ($||\xb'|| = ||\xb||$) and angles between vectors. Geometrically, this operation corresponds to a rotation, a reflection, or a combination thereof. It aligns the space with the principal axes (eigenvector directions).

    \item \textbf{Scaling ($\xb'' = \Lambda \xb'$):} Since $\Lambda$ is diagonal with eigenvalues $\lambda_i$ on the diagonal, this step scales the components along the new (eigenvector) axes. The $i$-th component is multiplied by $\lambda_i$. This is a stretching or compression along each principal axis.

    \item \textbf{Rotation/Reflection ($\yb = U \xb''$):} The matrix $U = (U^T)^{-1}$ is also orthogonal. It performs the inverse rotation/reflection, bringing the scaled vector back into the orientation of the original standard coordinate system.
\end{enumerate}

In essence, applying a symmetric matrix $A$ geometrically corresponds to \textbf{rotating/reflecting} the space to align with its natural scaling directions (eigenvectors), \textbf{scaling} along those directions, and then \textbf{rotating/reflecting} back.

\section{Properties from Diagonalization}

Diagonalization makes proving some important properties straightforward. If $A = P D P^{-1}$:

\begin{proposition}[Determinant and Trace]
\leavevmode
\begin{itemize}
    \item $\det(A) = \det(D) = \prod_{i=1}^n \lambda_i$. (The determinant is the product of the eigenvalues).
    \item $\tr(A) = \tr(D) = \sum_{i=1}^n \lambda_i$. (The trace is the sum of the eigenvalues).
\end{itemize}
\end{proposition}

\begin{proof}
\leavevmode
\begin{itemize}
    \item $\det(A) = \det(P D P^{-1}) = \det(P)\det(D)\det(P^{-1}) = \det(P)\det(D)\frac{1}{\det(P)} = \det(D)$.
    \item $\tr(A) = \tr(P D P^{-1}) = \tr(P^{-1} P D) = \tr(I D) = \tr(D)$ (using the cyclic property $\tr(XYZ) = \tr(ZXY)$).
\end{itemize}
The determinant and trace of a diagonal matrix are simply the product and sum of its diagonal entries (the eigenvalues), respectively.
\end{proof}

\section{Looking Ahead and Remarks}

\begin{itemize}
    \item \textbf{Projection Matrices:} We briefly mentioned that projection matrices have eigenvalues of only 0 and 1. We need to explore this topic further, likely during the Passover break.
    \item \textbf{Uniqueness:} The uniqueness of coordinate vectors for a given basis is fundamental. It's why we can equate matrix representations if they produce the same output coordinates for all input coordinates.
    \item \textbf{Action Item:} Please work through Question 4 from the current assignment. Consider how the concepts we discussed today might apply.
\end{itemize}

Keep thinking about these connections! The interplay between abstract transformations, concrete matrices, changes in perspective (bases), and finding simplifying structures (diagonalization) is central to the power and elegance of linear algebra.

\end{document}
% --- Document End ---